%%% LITERATUR
\usepackage[%
	backend=biber,% biber ist neuer und offeriert mehr Stile beim zitieren
	style=numeric,% Jedem Eintrag wird eine Nummer zugewiesen und so wird zitiert
	sorting=none,% Literatur wird in der Zitierreihenfolge gelistet
	citestyle=numeric-comp,% kompaktes numerisches zitieren 1-3 statt 1,2,3	
	backref=true% ermöglicht links vom literaturverzeichnis zurück
	]%
	{biblatex}% Bibliographie-Management-Paket
\addbibresource{./frontbackmatter/literatur.bib}% Dateipfad der Literatur-Datei
%
%
%%% ZITIEREN
\usepackage[%
	autostyle=true,% adaptiert den Quotation-Style durchgehend anhand der aktuell benutzten Sprache
	german=guillemets% Deutsche Sprache - entweder quotes (Anführungszeichen) oder guillemets (Anführungsstriche) 
	]{csquotes}% Paket für sinnvolles Zitieren!
\MakeAutoQuote{»}{«}% Zitierbefehle neu definieren, Shift+F1 und Shift+F2
\MakeAutoQuote*{›}{‹}% Shift+F3 und Shift+F4
\SetCiteCommand{\autocite}%
%
%
%%% VERLINKEN
\usepackage[% sollte immer nach Sprachen und Biliographie geladen werden!
	bookmarks=true,% Bookmarks einschalten
	bookmarksnumbered=true,% Erzeugt automatisch Links von Chapter, Section usw.
	pagebackref=false,% funktioniert nicht mit biblatex - aber eine ähnliche option ist dort eingestellt
	hypertexnames=true,% unbedingt auf true, damit biblatex zurückreferenzieren kann auf die richtige Stelle! falsche Verlinkungen zu Abbildungen usw. sollten durch besseres referenzieren gelöst werden und nicht durch Verstellen dieser option!
	breaklinks=true,% Links können getrennt werden
	linkbordercolor={0 0 1},% farbe der Links festlegen. Hier: Blaue links
	pdfborder={0 0 1}]% Stärke der Linkkästchen festlegen
	{hyperref}% Mit dem Paket hyperref können Links innerhalb des Dokumentes im späteren PDF Format angeklickt werden und schicken einen durch das PDF, so wie es sein soll. Tolles Feature, sollten alle PDFs haben. Zusätzlich können für die verschiedenen Reader noch .. definiert werden
% siehe http://www.pa.op.dlr.de/~PatrickJoeckel/pdflatex/index.html für eine tolle Hilfe
% Angaben in die PDF-Infos uebernehmen für PDF Reader
\makeatletter%
\hypersetup{% PDF-Dokument-Informationen setzen
            pdftitle={\@title},% Titel der Arbeit
            pdfauthor={\@author},% Author
            %pdfkeywords={},% CR-Klassifikation und ggf. weitere Stichworte
            %pdfsubject={}%
}%
\makeatother%
%%%%%%%%%%%%%%%%%%%%%%%%
\usepackage{bookmark}% "Adding package bookmark improves bookmarks handling. More features and faster updated bookmarks" - sollte nach hyperref geladen werden
%%%%%%%%%%%%%%%%%%%%%%%%
\usepackage[all]{hypcap}% Damit springt Hyperref nicht zu den Captions, sondern zum Start des Bildes/der Tabelle - muss nach hyperref geladen werden