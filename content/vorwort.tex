\chapter{Quick and dirty - Vorwort}%
\label{chap: vorwort}%
Hier findet man keine Installationsanweisungen, sondern nur alles zu den Befehlen in LaTex. Ihr habt noch nichts installiert? In der \textsc{readme.md} findet ihr alle Anweisungen zum Kaltstart.%


Ein paar kleine Dinge solltet ihr einstellen, \emph{bevor} ihr anfangt:%
\begin{itemize}%
\item Wollt ihr ein einseitige oder zweiseitige Abschlussarbeit? Wenn ihr euch für den beidseitigen Druck entschieden habt, dann stellt ihr in der Datei \textsc{/style/form-und-aufbau.tex} in \textsc{Zeile 3} \emph{twoside=true}.%
\item Wie groß ist eure Bindekorrektur? Fragt in der Druckerei nach und stellt in \textsc{Zeile 71} in demselben Dokument den Wert unter \emph{bindingoffset} entsprechend ein.%
\end{itemize}%
Anschließend tragt ihr in der Datei \textsc{/content/titlepage.tex} in \textsc{Zeile 1-8} euren Namen usw. ein. Dann könnt ihr mit eurer eigentlichen Abschlussarbeit starten.%

Extrem Ungeduldige und erfahrenere Latex-Benutzer finden in der pdf \textsc{quickanddirty-cheatsheet} die wichtigsten Befehle und legen direkt los. Alle anderen lesen jetzt weiter.%
\newpage