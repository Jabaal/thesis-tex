\chapter{Struktur und Texteigenschaften}%
\section{Allgemeiner Aufbau}%
Folgender Grundsätzlicher Aufbau der Diplomarbeit wird gefordert - von den optionalen Seiten laut Leitfaden optional ist nur das "Spezifische Titelblatt" übernommen worden:%
\begin{itemize}%
 \item Leerseite%
 \item Aufgabenblatt der Hochschule (1x original, 1x Kopie) \emph{wird hier nicht eingefügt, da diese Kopie in der Druckerei eingefügt werden muss!}%
 \item Spezifisches Titelblatt%
 \item Inhaltsverzeichnis%
 \item Abkürzungsverzeichnis%
 \item Verzeichnis der Formelzeichen und Symbole%
 \item Abbildungsverzeichnis%
 \item Tabellenverzeichnis%
 \item Inhalt des Dokuments%
 \item Literatur- und Quellenverzeichnis%
 \item Eidesstattliche Erklärung%
 \item Anlagenverzeichnis%
 \item Anlagen%
 \item Leerseite%
\end{itemize}%
%
%
%
\section{Dokumentstruktur}%
Oft will man Kapitel oder andere Teile der Arbeit umstrukturieren. Dafür empfiehlt es sich (weniger Scrollarbeit, eure Maushand wird es euch danken), zumindest die einzelnen Kapitel (eventuell mehr) in \emph{einzelne} Dateien zu packen, die man in einer Hauptdatei zusammenfügt. Einfügen solltet ihr \textsc{NUR} wie  in \ref{tab: einfuegen} zu sehen vornehmen. Der Befehl \textbackslash input\{\} wirkt dabei so, als ob ihr den Code direkt in euer Dokument einfügen würdet. %
\begin{table}[h]%
\begin{tabu} to \textwidth {X[l]X[l]}%
\toprule%
\textbackslash input\{./content/kapitelname.tex\} & Fügt den Tex-Code aus der Datei "`kapitelname.tex"' genau an der Stelle ein, wo der Befehl steht. Dabei liegen die Dateien im Ordner "`/content/"' in eurem Ordner mit der Datei \emph{template-htw-abschlussarbeit.tex}\\%
\bottomrule%
\end{tabu}%
\caption{Einfuegen}%
\label{tab: einfuegen}%
\end{table}%
%
%
\section{Kapitelstruktur}
Mit folgenden Befehlen aus \ref{tab: kapitel} Strukturiert man seine Überschriften. Diese 6 Ebenen solltet ihr maximal verwenden:
\begin{table}[h]
\begin{tabu} to \textwidth {X[l]X[l]}
\toprule
\textbackslash chapter\{Name\} & 1. Name\\
\textbackslash section\{Name\} & 1.1. Name\\
\textbackslash subsection\{Name\} & 1.1.1 Name \\
\textbackslash subsubsection\{Name\} & 1.1.1.1 Name \\
\textbackslash paragraph\{Name\} & 1.1.1.1.1 Name \\
\textbackslash subparagraph\{Name\} & 1.1.1.1.1.1 Name \\
\bottomrule
\end{tabu}
\caption{Kapitel}
\label{tab: kapitel}
\end{table}
Anmerkungen zur Struktur:
\begin{itemize}
\item Außer die voreingestellten nicht nummerierten Kapiteln werden alle Kapitel nummeriert in einer Abschlussarbeit!
\item Ein Unterpunkt benötigt mindestens zwei Einträge! Auf \textbackslash chapter\{\} müssen also mindestens zwei \textbackslash section\{\} folgen, damit ein Einfügen von \textbackslash section\{\} berechtigt ist!
\item Änderungen der Kapitelstruktur sind \emph{sofort} im linken Übersichtsbereich von \emph{TexMaker} unter \textsc{structure} sichtbar, yay!
\item Bei Änderungen der Kapitelstruktur muss man \emph{zwei Mal kompilieren}, damit man die Änderung auch im PDF sieht!
\end{itemize}
%
%
%
\section{Absätze}
Die wichtigsten Strukturen für eine Abschlussarbeit sind die folgenden beiden:
%
%
%
\section{Akzente setzen}
Für längere Akzente benutzt man \textit{kursive Schrift}. Funktioniert wie hier in \ref{tab: kursiv} zu sehen:\\
\begin{table}[h]
\begin{tabu} to \textwidth {X[l]X[l]}
\toprule
\textbackslash textit\{kursiv\} & \textsc{kursiv} \\
\bottomrule
\end{tabu}
\caption{kursiv}
\label{tab: kursiv}
\end{table}
%
%
Weil \textbf{fetter Text} mit der Schriftart \emph{Arial} hässlich aussieht und \underline{unterstreichen} mindestens genauso schäbig aussieht, verwenden wir weder fetten Text noch unterstreichen wir Wörter. Stattdessen nutzen wir \textsc{Kapitälchen} und \emph{Akzente}. \textsc{Kapitälchen funktionieren} wie in \ref{tab: kapitälchen} zu sehen:\\
\begin{table}[h]
\begin{tabu} to \textwidth {X[l]X[l]}
\toprule
\textbackslash textsc\{Kapitälchen\} & \textsc{Kapitälchen} \\
\bottomrule
\end{tabu}
\caption{Kapitälchen}
\label{tab: kapitälchen}
\end{table}
%
%
Um einzelne \emph{Akzent} zu setzen, nutzt man den Befehl aus \ref{tab: akzent}. Man beobachte, dass der \emph{Akzent} in kursiver Schriftumgebung anders gesetzt wird:
\begin{table}[h]
\begin{tabu} to \textwidth {X[l]X[l]}
\toprule
\textbackslash emph\{Akzent\} & \emph{Akzent} \\
Ein \emph{Akzent} in normaler Schriftumgebung & \textit{Ein \emph{Akzent} in kursiver Schriftumgebung}\\
\bottomrule
\end{tabu}
\caption{Akzent}
\label{tab: akzent}
\end{table}
%
%