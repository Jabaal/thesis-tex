\chapter{Struktur und Texteigenschaften}%
\section{Allgemeiner Aufbau}%
Folgender Grundsätzlicher Aufbau der Diplomarbeit wird gefordert - von den optionalen Seiten laut Leitfaden optional ist nur das "Spezifische Titelblatt" übernommen worden:%
\begin{itemize}%
 \item Leerseite%
 \item Aufgabenblatt der Hochschule (1x original, 1x Kopie) \emph{wird hier nicht eingefügt, da diese Kopie besser in der Druckerei eingefügt werden sollte!}%
 \item Spezifisches Titelblatt%
 \item Inhaltsverzeichnis%
 \item Abkürzungsverzeichnis%
 \item Verzeichnis der Formelzeichen und Symbole%
 \item Abbildungsverzeichnis%
 \item Tabellenverzeichnis%
 \item Inhalt des Dokuments%
 \item Literatur- und Quellenverzeichnis%
 \item Eidesstattliche Erklärung%
 \item Anlagenverzeichnis%
 \item Anlagen%
 \item Leerseite%
\end{itemize}%
%
%
\section{Allgemeine Texteingabe}%
Latex setzt den Text besser als die meisten Programme. \href{https://tex.stackexchange.com/questions/110133/visual-comparison-between-latex-and-word-output-hyphenation-typesetting-ligat#110140}{Hier} und \href{http://www.rtznet.nl/zink/latex.php}{hier} sind Beispiele dazu.%

Damit auch keine Fehler passieren, sollte man am Ende einer Zeile immer ein Prozentzeichen "\%" setzen. Die Begründung dazu findet man \href{https://tex.stackexchange.com/questions/7453/what-is-the-use-of-percent-signs-at-the-end-of-lines#7459}{hier}.%

Falls man also innerhalb eines Kapitels eine Einrückung haben möchte, muss man nur eine Leerzeile(siehe \ref{tab: leerzeile}) einfügen.\\%
\begin{table}[ht]%
\begin{tabu} to \textwidth {X[c]X[c]}%
\toprule%
	\begin{minipage}[t]{0.4\textwidth}% minipage wird genutzt, damit die Befehle ohne Einrückung geschrieben werden können
	Erste Zeile Latex Code\\%
	\\%
	Zweite Zeile Latex Code%
	\end{minipage}%
&%
	\begin{minipage}[t]{0.4\textwidth}%
	Erste Zeile Latex Code\\%
	\hspace*{10.95pt}Zweite Zeile Latex Code% indent setzen - durch \the\parindent kann man die aktuelle Größe im Dokument ins PDF schreiben lassen
	\end{minipage}\\%
\bottomrule%
\end{tabu}%
\caption{Einrückung durch Leerzeile}%
\label{tab: leerzeile}%
\end{table}\\%
%
Einen erzwungenen Umbruch sollte man nur im äußersten Notfall gebrauchen! Dieser wird wie in \ref{tab: backslash} erzwungen, erstellt eine neue Zeile \emph{(!)} und bricht den Blocksatz ab \emph{(!)}.\\%
\begin{table}[h]%
\begin{tabu} to \textwidth {X[c]X[c]}%
\toprule%
	\begin{minipage}{0.4\textwidth}%
	Erste Zeile Latex Code\textbackslash \textbackslash Zweite Zeile Latex Code
	\end{minipage}%
&%
	\begin{minipage}{0.4\textwidth}%
	Erste Zeile Latex Code\\%
	\noindent Zweite Zeile Latex Code%
	\end{minipage}\\%
\bottomrule%
\end{tabu}%
\caption{Erzwungener Zeilenumbruch}%
\label{tab: backslash}%
\end{table}\\%
Befolgt man diese Ratschläge, muss man sich so gut wie keine Gedanken machen, um die Formatierung.
%
%
\section{Dokumentstruktur}%
Oft will man Kapitel oder andere Teile der Arbeit umstrukturieren. Dafür empfiehlt es sich zumindest die Kapitel in \emph{einzelne} Dateien zu packen, die man in einer Hauptdatei zusammenfügt. Einfügen solltet ihr \textsc{NUR} nach dem Muster in Tabelle \ref{tab: einfuegen}. Der Befehl \textbackslash input\{\} wirkt dabei so, als ob ihr den Code direkt in euer Dokument einfügen würdet.%
\begin{table}[h]%
\begin{tabu} to \textwidth {X[c]X[c]}%
\toprule%
\textbackslash input\{./content/kapitelname.tex\} & Fügt den Tex-Code aus der Datei "kapitelname.tex" genau an der Stelle ein, wo der Befehl steht. Dabei liegen die Dateien im Ordner "/content/" in dem Ordner mit der Datei \emph{main-htw-abschlussarbeit.tex}\\%
\bottomrule%
\end{tabu}%
\caption{Dokumentstruktur durch Einfügen}%
\label{tab: einfuegen}%
\end{table}%
%
%
\section{Kapitelstruktur}%
Mit folgenden Befehlen aus \ref{tab: kapitelstruktur} Strukturiert man seine Überschriften. Diese 6 Ebenen solltet ihr maximal verwenden.\\%
\begin{table}[h]%
\begin{tabu} to \textwidth {X[c]X[c]}%
\toprule%
\textbackslash chapter\{Name\} & 1. Name\\%
\textbackslash section\{Name\} & 1.1. Name\\%
\textbackslash subsection\{Name\} & 1.1.1 Name \\%
\textbackslash subsubsection\{Name\} & 1.1.1.1 Name \\%
\textbackslash paragraph\{Name\} & 1.1.1.1.1 Name \\%
\textbackslash subparagraph\{Name\} & 1.1.1.1.1.1 Name \\%
\bottomrule%
\end{tabu}%
\caption{Kapitelstruktur}%
\label{tab: kapitelstruktur}%
\end{table}%
Anmerkungen zur Struktur:\\%
\begin{itemize}%
\item Außer den voreingestellten nicht nummerierten Kapiteln werden alle Kapitel nummeriert in einer Abschlussarbeit! Eine Ausnahme bildet der Anhang - weitere Informationen findet ihr im \ref{chap: anhang}%
\item Ein Unterpunkt benötigt mindestens zwei Einträge! Auf \textbackslash chapter\{\} müssen also mindestens zwei \textbackslash section\{\} folgen, damit ein Einfügen von \textbackslash section\{\} berechtigt ist!%
\item Änderungen der Kapitelstruktur sind \emph{sofort} im linken Übersichtsbereich von \emph{TexMaker} unter \textsc{structure} sichtbar, yay!%
\item Bei Änderungen der Kapitelstruktur muss man \emph{zwei Mal kompilieren}, damit man die Änderung auch im PDF sieht!%
\end{itemize}%
%
%
\section{Akzente setzen}%
Für längere Akzente benutzt man \textit{kursive Schrift}. Funktioniert wie hier in \ref{tab: kursiv} zu sehen:%
%
\begin{table}[h]%
\begin{tabu} to \textwidth {X[c]X[c]}%
\toprule%
\textbackslash textit\{kursiv\} & \textsc{kursiv} \\%
\bottomrule%
\end{tabu}%
\caption{Kursiv}%
\label{tab: kursiv}%
\end{table}%
%
Weil \textbf{fetter Text} mit der Schriftart \emph{Arial} innerhalb eines Textes wirklich fies aussieht und \underline{unterstreichen} mindestens ebenso schäbig aussieht, verwenden wir weder fetten Text noch unterstreichen wir Wörter. Stattdessen nutzen wir \textsc{Kapitälchen} und \emph{Akzente}. \textsc{Kapitälchen funktionieren} wie in \ref{tab: kapitälchen} zu sehen:\\%
\begin{table}[h]%
\begin{tabu} to \textwidth {X[c]X[c]}%
\toprule%
\textbackslash textsc\{Kapitälchen\} & \textsc{Kapitälchen} \\%
\bottomrule%
\end{tabu}%
\caption{Kapitälchen}%
\label{tab: kapitälchen}%
\end{table}%
%
Um einzelne \emph{Akzent} zu setzen, nutzt man den Befehl aus \ref{tab: akzent}. Man beobachte, dass der \emph{Akzent} in kursiver Schriftumgebung anders gesetzt wird:%
\begin{table}[h]%
\begin{tabu} to \textwidth {X[c]X[c]}%
\toprule%
\textbackslash emph\{Akzent\} & \emph{Akzent} \\%
Ein \emph{Akzent} in normaler Schriftumgebung & \textit{Ein \emph{Akzent} in kursiver Schriftumgebung}\\%
\bottomrule%
\end{tabu}%
\caption{Akzente}%
\label{tab: akzent}%
\end{table}%
%
Für Tabellenüberschriften benötigt man gelegentlich doch eine fette Schriftart. In Tabelle \ref{tab: fett} findet man den Befehl dazu. Kapitelüberschriften und einige andere Formatierungen sind übrigens ebenso von dieser Regel ausgenommen.\\%
\begin{table}[h]%
\begin{tabu} to \textwidth {X[c]X[c]}%
\toprule%
\textbackslash textbf\{fett\} & \textbf{fett} \\%
\bottomrule%
\end{tabu}%
\caption{Fette Schrift}%
\label{tab: fett}%
\end{table}%
%
%
\section{Besondere Zeichen}%
Quotation, vorher festlegen mit csquotes, da sonst Fehler entstehen\\%
wichtige sonderzeichen listen (zum beispiel textbackslash usw.)\\%
Viele andere Zeichen, die nur in der Mathematik-Umgebung funktionieren, findet ihr in TexMaker auf der linken Seite unter den Zeichen \emph{Relation symbols} (Symbol: $\div$), \emph{Arrow symbols} (Symbol: $\Rightarrow$), \emph{Miscellaneous symbols} (Symbol: $\forall$), \emph{Delimiters} (Symbol: $\lbrace\rbrace$) und \emph{Greek letters} (Symbol: $\lambda$).%