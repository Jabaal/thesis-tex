\chapter{Quick and dirty - Vorwort}%
\label{chap: vorwort}%
Hier findet man keine Installationsanweisungen, sondern nur alles zu den Befehlen in LaTex. Ihr habt noch nichts installiert? In der \textsc{readme.md} findet ihr alle Anweisungen zum Kaltstart.%


Ein paar kleine Dinge solltet ihr einstellen, \emph{bevor} ihr anfangt:%
\begin{itemize}%
\item Wollt ihr ein einseitige oder zweiseitige Abschlussarbeit?%

Wenn ihr euch für den beidseitigen Druck entschieden habt, dann stellt ihr in der Datei \textsc{/style/form-und-aufbau.tex} in \textsc{Zeile 3} \emph{twoside=true}.%
\item Wie groß ist eure Bindekorrektur?%

Fragt in der Druckerei nach und stellt in \textsc{Zeile 51} in demselben Dokument den Wert unter \emph{bindingoffset} entsprechend ein.%
\item Falls ihr keine exakten Vorgaben habt, wie groß der Textbereich sein soll, den ihr beschreiben dürft, dann kommentiert unbedingt in diesem Dokument \textsc{Zeile 45 ff} aus und nutzt die automatische Berechnung mittels \emph{KOMA-Script}.%

\textsc{Zeile 5} muss einkommentiert werden, damit die Satzspiegelberechnung automatisch stattfindet und die Bindekorrektur findet man dann in  \textsc{Zeile 6} unter \emph{BCOR}. Im Dokument \textsc{/style/pakete.tex} muss man anschließend noch die \textsc{allerletzte Zeile} einkommentieren und schon hat man einen angenehmeren Satzspiegel.%
\end{itemize}%
Anschließend tragt ihr in der Datei \textsc{/content/titel.tex} in \textsc{Zeile 4-8} euren Namen usw. ein. Dann könnt ihr mit eurer eigentlichen Abschlussarbeit starten.%

Extrem Ungeduldige und erfahrenere Latex-Benutzer finden in der Datei \textsc{quickndirty-cheatsheet.pdf} die wichtigsten Befehle und legen direkt los. Alle anderen lesen jetzt weiter.%
\newpage