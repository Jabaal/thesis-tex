%%% FORMAT UND AUFBAU
%%% ALLGEMEINE FORMATIERUNGEN
\documentclass[
	twoside=false, % Doppelseitig schreiben oder nicht. Wichtig für automatische Bindekorrektur (BCOR).
	paper=a4, % Papiergröße festlegen, hier A4
	DIV=13, % Mit Hilfe der Option DIV=Faktor wird festgelegt, in wie viele Streifen die Seite horizontal und vertikal bei der Satzspiegelkonstruktion eingeteilt wird. DIV=calc nimmt einem diese Arbeit ab (hängt ja auch von der Schriftart und -größe ab usw.)
	BCOR=10mm, % Mit Hilfe der Option BCOR=Korrektur geben Sie den absoluten Wert der Bindekorrektur an, also die Breite des Bereichs, der durch die Bindung von der Papierbreite verloren geht. Am besten in der Druckerei nachfragen! Man kann einen Wert von ca. 10mm für Arbeiten bis 100 Seiten annehmen.
	parskip=false, % Wie Absätze gekennzeichnet werden. Zitat: "parskip=false Absätze werden durch einen Einzug der ersten Zeilen von einem Geviert (1 em) gekennzeichnet. Der erste Absatz eines Abschnitts wird nicht eingezogen." Alles andere ist hier nicht sinnvoll, siehe scrguide.pdf und Leitfaden_fuer_wiss_Dokumente.pdf!
	fontsize=11pt, % Schriftgröße
	listof=totoc, % Abbildungsverz. usw. werden in das Inhaltsverzeichnis aufgenommen
	bibliography=totoc % % Literaturverzeichnis wird in das Inhaltsverzeichnis aufgenommen
	]{scrreprt}
%
%
%%% HEADER und FOOTER DESIGN
\usepackage[ % Header und Footer designen
	autooneside=true, % ???
	headsepline=true, % Trennungsstrich für den Header (außer bei neuen Kapiteln und bei den Indizes)
	footsepline=true, % Trennungsstrich für den Footer (außer bei neuen Kapiteln und bei den Indizes)
	plainfootsepline=true % Trennungsstrich für den Footer bei neuen Kapiteln und bei den Indizes
	]{scrlayer-scrpage}
%
%
%%% WAS steht WO im HEADER und FOOTER
\pagestyle{scrheadings} % scrheadings einstellen, damit man alles weitere im Header und Footer richtig definieren kann
\clearscrheadfoot % Header und footer clearen (sonst hat man die Seitenzahl eventuell zwei mal im Footer...)
\automark[section]{chapter} % Von den ersten beiden Gliederungsebenen jeweils die niedrigste im Header anzeigen
\chead{\headmark} % Zentrierte Überschrift im Header
 \ofoot[\pagemark]{\pagemark}  % Seitenzahl für einseitigen Druck im Footer auf der rechten Seite platzieren, für zweiseitigen Druck immer außen platzieren
\addtokomafont{pagefoot}{\small} % Fußnoten etwas kleiner als die normale Schriftgröße setzen - zur Sicherheit den Befehl wiederholen - falls sich die Standard-Einstellungen ändern (safety first!)
%
%
%%% SPRACHE
\usepackage{polyglossia} % statt Babel nutzt man für xelatex polyglossia!
\setmainlanguage[
	variant=german, %es geht auch austrian oder swiss
	spelling=new, %neue Rechtschreibung ein/aus
	latesthyphen=true, %ausschalten bei bugs - neueste hyphenation
	babelshorthands=false %Trennung wie beim grauenvollen Paket Babel ausschalten		
	]%
{german}%
%
%
%%% SCHRIFTART
\usepackage{fontspec} % "Normale" Schriftarten können verwendet werden (die, die auf dem PC selbst installiert sind)
\setmainfont{Arial} % Schriftart festlegen
%
%
%%% SPRACHE mit babel
% \usepackage[ngerman]{babel} % Für richtige Silben- und Worttrennung in der jeweiligen Sprache und die richtige Sprache in den automatisch generierten Verzeichnissen wird das Paket babel genutzt.
% Paket babel ngerman für die "neue" deutsche Sprache nutzen - eingeführt 1998, es ist 2018!
% siehe: https://tex.stackexchange.com/questions/67549/whats-the-difference-between-ngerman-and-german-in-babel
%
%
%%% ZEILENABSTAND
\usepackage{setspace} % Zeilenabstand einstellen mit setspace.sty - so werden auch in Fußzeilen die alten Abstände beibehalten! Siehe Sündenkatalog
\onehalfspacing % 1,5 Zeilig gewählt
% oder, falls dieses Paket mal nicht funktioniert:
% \linespread{1.2} % für Werte schaut man hier: https://www.sharelatex.com/learn/Paragraph_formatting#Line_spacing
%
%
%%% LEERE SEITEN
%\usepackage{afterpage} % Paket für eine leere Seite laden, damit unser Dokument genau so aussieht, wie gefordert (insbesondere nötig für zweiseitigen Druck!!!
% siehe https://tex.stackexchange.com/questions/36880/insert-a-blank-page-after-current-page
%\newcommand\blankpage{ % Befehl neu definieren - leere Seite ohne Header und Footer und den Counter für das Seitenzählen zurücksetzen
	%\null
    %\thispagestyle{empty}%
    %\addtocounter{page}{-1}%
    %\newpage}% Alles was mit dem Format und Aufbau zu tun hat, ist hier zu finden
%%% LADEN UND KONFIGURIEREN ALLER BENÖTIGTEN PAKETE
%%% LITERATUR
\usepackage[%
	backend=biber,% biber ist neuer und offeriert mehr Stile beim zitieren
	style=numeric,% Jedem Eintrag wird eine Nummer zugewiesen und so wird zitiert
	sorting=none,% Literatur wird in der Zitierreihenfolge gelistet
	citestyle=numeric-comp,% kompaktes numerisches zitieren 1-3 statt 1,2,3	
	backref=true% ermöglicht links vom literaturverzeichnis zurück
	]%
	{biblatex}% Bibliographie-Management-Paket
\addbibresource{../frontbackmatter/literatur.bib}% Dateipfad der Literatur-Datei
%
%
%%% ZITIEREN
\usepackage{csquotes}% Paket für sinnvolles Zitieren!
%
%
%%% VERLINKEN
\usepackage[% sollte immer nach Sprachen und Biliographie geladen werden!
	bookmarks=true,% Bookmarks einschalten
	bookmarksnumbered=true,% Erzeugt automatisch Links von Chapter, Section usw.
	pagebackref=false,% funktioniert nicht mit biblatex - aber eine ähnliche option ist dort eingestellt
	hypertexnames=true,% unbedingt auf true, damit biblatex zurückreferenzieren kann auf die richtige Stelle! falsche Verlinkungen zu Abbildungen usw. sollten durch besseres referenzieren gelöst werden und nicht durch Verstellen dieser option!
	breaklinks=true,% Links können getrennt werden
	linkbordercolor={0 0 1},% farbe der Links festlegen. Hier: Blaue links
	pdfborder={0 0 1}]% Stärke der Linkkästchen festlegen
	{hyperref}% Mit dem Paket hyperref können Links innerhalb des Dokumentes im späteren PDF Format angeklickt werden und schicken einen durch das PDF, so wie es sein soll. Tolles Feature, sollten alle PDFs haben. Zusätzlich können für die verschiedenen Reader noch .. definiert werden
% siehe http://www.pa.op.dlr.de/~PatrickJoeckel/pdflatex/index.html für eine tolle Hilfe
% Angaben in die PDF-Infos uebernehmen für PDF Reader
\makeatletter%
\hypersetup{% PDF-Dokument-Informationen setzen
            pdftitle={\@title},% Titel der Arbeit
            pdfauthor={\@author},% Author
            %pdfkeywords={},% CR-Klassifikation und ggf. weitere Stichworte
            %pdfsubject={}%
}%
\makeatother%
%
%
\usepackage{bookmark}% "Adding package bookmark improves bookmarks handling. More features and faster updated bookmarks" - sollte nach hyperref geladen werden%
%%% ABKÜRZUNGEN
% Hier stehen alle Abkürzungen
\newacronym{slub}{SLUB}{Sächsische Landesbibliothek - Staats- und Universitätsbibliothek Dresden}
\newacronym{htw}{HTW}{Hochschule für Technik und Wirtschaft Dresden}%
%%% FORMELZEICHEN UND SYMBOLE
% Hier die Formelzeichen einfügen in der Form
%\newglossaryentry{<label>}{name={<name>},type=symbols,symbol={<symbol>},description={<Description>}}%
	%<label> Das Label
	%<name> Der Name/die Bezeichnung des Symbols - z.B. Masse (für Symbol m)
	%<symbol> Das Symbol - aus \ensuremath{\alpha} wird das Symbol für alpha
	%<Description> Die Beschreibung des Symbols/Formelzeichens eventuell die Einheit noch hinzufügen
%mit \gls{<label>} referenziert man den <name>
%mit \glssymbol{<label>} referenziert man das <symbol>%
%%%%%%%%%%%%%%%%%%%%%%%%%%%%%%%%%%%%%%%%%%%%%%%%%%%%%%%%%%%%%
\begin{document}%
%%% SPEZIFISCHES TITELBLATT
\titlehead{\centering\includegraphics[width=7cm]{htw-logo}}% Bild des HTW-Logos
\subject{Hochschule für Technik und Wirtschaft Dresden\\%
		Fakultät hier einfügen}% Noch die Fakultät einfügen
\title{Diplomarbeitsthema hier einfügen}% Thema hier einfügen, wird in die Eidesstattliche Erklärung übernommen
\subtitle{Diplom-/Bachelor-/Masterarbeit}% Um was für eine Abschlussarbeit handelt es sich?
\author{Hier könnte dein Name stehen}% eigenen Namen einfügen, dieser wird automatisch in die Eidesstattliche Erklärung übernommen
\date{\today}% heutiges Datum
\publishers{Betreuer usw.}% Hier die Betreuer usw. einfügen
%\thanks{Fußnote}% hier ausgeblendet, da nicht sinnvoll
\maketitle% Erstellt das Titelblatt mit den vorher eingegebenen Daten
\newpage% Seitenumbruch%
%%% VERZEICHNISSE
%%% INHALTSVERZEICHNIS
\pagenumbering{Roman}% Seitenzahlen mit großen römischen Buchstaben beginnen
\pdfbookmark[section]{\contentsname}{toc} %Fügt in der Navigation (und nur in der PDF-Datei!) den Eintrag Inhaltsverzeichnis hinzu
\tableofcontents% Inhaltsverzeichnis erstellen
\newpage% 
%
%
%%% ABKÜRZUNGSVERZEICHNIS
\setglossarystyle{long3col}% Stil festlegen
\printnoidxglossaries%
\newpage%
%
%
% Verzeichnis der Formelzeichen und Symbole
Verzeichnis der Formelzeichen und Symbole%
Hier muss unbedingt noch mit Glossaries gearbeitet werden. nomencl ist einfach zu schwach.%
\newpage%
%
%
%%% ABBILDUNGSVERZEICHNIS
\listoffigures% Erstellt das Abbildungsverzeichnis
\newpage%
%
%
%%% TABELLENVERZEICHNIS
\listoftables% Erstellt das Tabellenverzeichnis
\newpage%%
%%% INHALT
\cleardoubleoddplainpage% Bei Zweiseitigem Druck wird so IMMER der Inhalt auf der rechten Seite gestartet und falls nötig eine Seite ohne Titel, aber mit Fuß eingefügt. Sieht einfach besser aus. \cleardoubleoddemptypage fügt leere Seite ein.
\pagenumbering{arabic}% hier beginnt die Zählung der Seiten in arabischen Zahlen
\chapter{Quick and dirty - Vorwort}%
Ein paar kleine Dinge solltet ihr einstellen, bevor ihr drucken geht:%
\begin{itemize}%
\item Einseitig oder zweiseitig bedrucken? Wenn ihr euch für den beidseitigen Druck entschieden habt, dann ändert den Wert von \emph{twoside} in der Datei /style/form-und-aufbau.tex in \textsc{Zeile xxx} auf \emph{true}. Dann noch in \textsc{Zeile xxx} autooneside auf \emph{false} stellen, sonst funktionieren die Header nicht wie gewünscht.
\item Wie groß ist eure Bindekorrektur? Fragt in der Druckerei nach und ändert den Wert \emph{BCOR} in \textsc{Zeile xxx} in demselben Dokument entsprechend.%
\end{itemize}%
Als erste Amtshandlung solltet ihr euren Namen usw. auf der Titelseite eintragen - in der Datei content/titlepage.tex in \textsc{Zeile xx-xx} findet ihr diese Felder. Anschließend könnt ihr starten.% 
Extrem ungeduldige nutzen die PDF \emph{quickanddirty-cheatsheet.pdf} und überspringen die folgenden kurzen Hinweise, warum wie formatiert wird, um direkt loszulegen.\\%%
%%% LITERATUR- UND QUELLENVERZEICHNIS
\cleardoubleoddplainpage% Titel, aber nicht Fuß unterdrückt - und eine Leere Seite vor der Bibliographie, so dass diese Rechts startet
\printbibliography[% Erstellt das Verzeichnis
	heading=bibintoc% Die Überschrift zum Inhaltsverzeichnis hinzufügen
	]%
\newpage%
%%% EIDESSTATTLICHE ERKLÄRUNG
% Eidesstattliche Erklärung
\addcontentsline{toc}{chapter}{Erklärung über die eigenständige Erstellung der Arbeit}%
\chapter*{Erklärung über die eigenständige Erstellung der Arbeit}%
Hiermit erkläre ich, \makeatletter \@author \makeatother , dass ich die vorgelegte Arbeit mit dem Titel \makeatletter \textsf{\@title} \makeatother selbstständig verfasst, keine anderen als die angegebenen Quellen und Hilfsmittel benutzt sowie alle wörtlich oder sinngemäß übernommenen Stellen in der Arbeit als solche und durch Angabe der Quelle gekennzeichnet habe. Dies gilt auch für Zeichnungen, Skizzen, bildliche Darstellungen sowie für Quellen aus dem Internet.\\%
Mir ist bewusst, dass die Hochschule für Technik und Wirtschaft Dresden Prüfungsarbeiten stichprobenartig mittels der Verwendung von Software zur Erkennung von Plagiaten überprüft.\\%
Ferner gestatte ich der Hochschule für Technik und Wirtschaft Dresden, die beiliegende Diplomarbeit unter Beachtung insbesondere urheber-, datenschutz- und wettbewerbsrechtlicher Vorschriften für Lehre und Forschung zu nutzen.
Es ist mir bekannt, dass für die Weitergabe oder Veröffentlichung der Arbeit die Zustimmung der HTW Dresden sowie der an der Aufgabenstellung und Durchführung der Arbeit unmittelbar beteiligten Partnereinrichtungen erforderlich ist.\\%

\vfill%

Ort, \today \hspace{35mm} \hrulefill \\
\hspace*{83mm} Unterschrift
\newpage%
%%% ANLAGEN
\appendix% Anhang starten
\chapter*{Anlagen}% Überschrift Ohne Nummerierung hinzufügen
\label{chap: anhang}% Setzt ein Label für anlagen
\addcontentsline{toc}{chapter}{Anlagen}% Zum Inhaltsverzeichnis hinzufügen
Hier die Anlagen hinzufügen.

Falls eine Nummerierung mit Großbuchstaben für verschiedene Kategorien erforderlich ist, einfach die Zeile mit \emph{\textbackslash addcontentsline  ...} auskommentieren und den \emph{*} bei chapter entfernen.%
\newpage%%
%%%%%%%%%%%%%%%%%%%%%%%%%%%%%%%%%%%%%%%%%%%%%%%%%%%%%%%%%%%%%
\end{document}%
