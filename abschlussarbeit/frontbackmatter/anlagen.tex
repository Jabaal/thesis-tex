\appendix% Anhang starten
\chapter*{Anlagen}% Überschrift Ohne Nummerierung hinzufügen
\label{chap: anlagen}% Setzt ein Label für anlagen
\addcontentsline{toc}{chapter}{Anlagen}% Zum Inhaltsverzeichnis hinzufügen
Hier die Anlagen hinzufügen. Die Anforderungen sind sehr unterschiedlich. Muss noch überarbeitet werden.
\newpage%

\section*{Datei-/Ordnerstruktur}% Datei und Ordner-Struktur
\begin{forest}
      for tree={
%        font=\ttfamily,% legt die Schriftart fest
        grow'=0,%
        child anchor=west,%
        parent anchor=south,%
        anchor=west,%
        calign=first,%
        inner xsep=7pt,%
        edge path={%
          \noexpand\path [draw, \forestoption{edge}]%
          (!u.south west) +(7.5pt,0) |- (.child anchor) pic {folder} \forestoption{edge label};%
        },%
        % style for your file node 
        file/.style={edge path={\noexpand\path [draw, \forestoption{edge}]%
          (!u.south west) +(7.5pt,0) |- (.child anchor) \forestoption{edge label};},%
          inner xsep=2pt,
%          font=\small\ttfamily% legt die Schriftart für die Datei fest
                     },
        before typesetting nodes={
          if n=1
            {insert before={[,phantom]}}
            {}
        },
        fit=band,
        before computing xy={l=15pt},
      }
%%%%%%%%%%%%%%%%%%%%%%%%%%%%%%%%%%%%%%%%%%%%% Hier kommt die Datei-/Ordnerstruktur hin  
    [USB-Stick
      [Ordner 1
      ]
      [Ordner 2
        [Ordner 2.1
        ]
        [Ordner 2.2
        ]
        [Datei.pdf,file
        ]
      ]
      [Ordner 3
      ]
      [Ordner 4
      ]
    ]
 \end{forest}
\newpage
